\documentclass[12pt,hyperref={bookmarks=false}]{beamer}

%\usepackage[no-math]{fontspec} % https://www.ptt.cc/bbs/LaTeX/M.1505577244.A.952.html
\usepackage{xeCJK}
\usepackage{amsfonts}
\usepackage{amssymb}
\usepackage{amsmath}
%\usepackage{musixtex}
\usepackage{enumitem}
\usepackage{algorithm,algorithmic}


\renewcommand\qedsymbol{$\blacktriangleleft$}

%\setCJKmainfont{新細明體}
\setCJKmainfont{標楷體}

\usetheme{Malmoe}
\usecolortheme{dolphin}
\usefonttheme[onlymath]{serif}
%\usefonttheme{serif} % professionalfonts
% default, serif, professionalfonts, structurebold, structureitalicserif, structuresmallcapsserif
\useoutertheme{miniframes} %{infolines}
\usepackage{xmpmulti}

\linespread{1.2}

%\newenvironment{num}
%{\leftmargini=6mm\leftmarginii=8mm\begin{itemize}}
%{\end{itemize}}

%\newenvironment{enu}
%{\leftmargini=6mm\leftmarginii=8mm\begin{enumerate}}
%{\end{enumerate}}

%-------------------------------------------------------------
\title{離散數學 107-2}
\subtitle{Homework 08}
\author{姓名: 葉子瑄。學號: 107590041}
\date{Handout: 2019.05.21 (week-14)}

%-------------------------------------------------------------
% page number
% https://tex.stackexchange.com/questions/137022/how-to-insert-page-number-in-beamer-navigation-bars
%-------------------------------------------------------------

\setbeamertemplate{navigation symbols}
{ \insertslidenavigationsymbol 
  \insertframenavigationsymbol   
  \insertsubsectionnavigationsymbol  
  \insertsectionnavigationsymbol
  \insertdocnavigationsymbol  
  \insertbackfindforwardnavigationsymbol 
  \hspace{1em}  
  \usebeamerfont{footline} 
  \insertframenumber
  ~-~
  \inserttotalframenumber
}
\setcounter{page}{1} 
\pagenumbering{arabic} 


%----------------------------------------------------------------------------------
\begin{document}
%----------------------------------------------------------------------------------

\begin{frame}
\titlepage
\end{frame}

\raggedright

\begin{frame}
% \tiny \scriptsize \footnotesize \small \normalsize \large \Large \LARGE \huge \Huge
%\normalsize
\footnotesize
\tableofcontents
\end{frame}
	
%----------------------------------------------------------------------------------
\section{題目}
%----------------------------------------------------------------------------------

    %----------------------------------------------------------------------------------
	\subsection{題目與注意事項}
	%----------------------------------------------------------------------------------
	
	%----------------------------------------------------------------------------------
	\begin{frame}
	\frametitle{Homework 08 題目}
	\fontsize{10pt}{11pt}\selectfont
	\setlength{\baselineskip}{5pt}
	\begin{columns}
	\begin{column}{0.68\textwidth}
	\begin{enumerate}[label=(Prob. \arabic*)]
	\setlength\itemsep{0em}
	\item page 536, chapter 8.1 Exercise 2
	\item page 552, chapter 8.2 Exercise 32
	\item page 561, chapter 8.3 Exercise 8
	\item page 575, chapter 8.4 Exercise 4(a)
	\item page 584, chapter 8.5 Exercise 2
	\item page 591, chapter 8.6 Exercise 8


	\end{enumerate}
	\end{column}
	
%	\begin{column}{0.45\textwidth}
%	\begin{enumerate}[label=(Prob. \arabic*)]
%	\addtocounter{enumi}{21}
%	\setlength\itemsep{0em}
%	\item page xx, chapter xx Exercises xx	
%	\end{enumerate}
%	\end{column}
	
	\end{columns}
	\end{frame}
	
	%----------------------------------------------------------------------------------
	\begin{frame}
	\frametitle{注意事項}
	\fontsize{10}{10pt}\selectfont
	\begin{enumerate}[label=(\alph*)]
	\item 要熟悉 LaTeX 請翻閱 \ \htmladdnormallink{\color{blue}lshort}{https://ctan.org/tex-archive/info/lshort/}。
	\item 記得在最後一頁,回報\selectfont \color{red}{完成作業小時數}(估算,取整數)\selectfont \color{black}{。}
	\item 將檔案夾命名為 \texttt{hw08\_107820xxx},將檔案夾壓縮成 \texttt{hw08\_107820xxx.zip},上傳到\htmladdnormallink{\color{blue}網路學園}{http://elearning.ntut.edu.tw}。
	\item LaTeX 數學符號請查此表: \ \htmladdnormallink{\color{blue}List of LaTeX mathematical symbols}{https://oeis.org/wiki/List_of_LaTeX_mathematical_symbols}。
	\item 作業抄襲,以零分計。作業提供給他人抄襲,以零分計。
	\item 作業遲交一週內成績打五折,作業遲交超過一週以零分計。
	\end{enumerate}
	\end{frame}

%----------------------------------------------------------------------------------
\section{作答區}
%----------------------------------------------------------------------------------

    %----------------------------------------------------------------------------------
	\subsection{解題}
	%----------------------------------------------------------------------------------
	
	%----------------------------------------------------------------------------------
	\begin{frame}
	\frametitle{Problem 1 (8.1 Exercise 2)}
	\fontsize{10}{10pt}\selectfont
    (a) A permutation of a set with n elements consists of a choice of a first element (which can be done in n
ways), followed by a permutation of a set with n − 1 elements. \\
Therefore  P_0 = 1,P_1=1,Pn = nPn−1 ,~ &  when ~$n$\ge 2.\\
    
	\end{frame}
	%---------------------------------------------------------------------------------	
	\begin{frame}
	\frametitle{Problem 1 (8.1 Exercise 2)}
	\fontsize{10}{10pt}\selectfont
    \vspace{0.3cm}
	(b)\begin{eqnarray*}{\fontsize{10}{10pt}}
    P_{n} & = & nP_{n} - 1\\
	~ & = & n(n -1)P_{n-2}\\
	~ & = & n(n -1)(n - 2)P_{n-3}\\
	~ & = & .....\\
	~ & = & n(n -1)(n - 2)......(3)(2)P_{1}\\
	~ & = & n(n -1)(n - 2)......(3)(2)(1)\\
	~ & = & n!\\
    \end{eqnarray*}
	\end{frame}
	%----------------------------------------------------------------------------------
	\begin{frame}
	\frametitle{Problem 2 (8.2 Exercise 32)}
	\fontsize{10}{10pt}\selectfont
	\vspace{0.3cm}
	The associated homogeneous recurrence relation is a_{n} = 2a_{n−1} . \\
	We easily solve it to obtain a_{n}^{(h)} = α2_{n} .\\
	 Next we need a particular solution to the given recurrence relation.\\
	  By Theorem 6 we want to look for a function of the form a_{n} = cn·2_{n}.\\ We plug this into our recurrence relation and obtain cn \cdot 2^n & = & 2c(n−1)2^{n-1} + 3 \cdot 2^n .\\
We divide through by 2_{n−1}.\\
Obtaining 2cn = 2c(n − 1) + 6, whence with a little simple algebra \\c = 3. $a_{n} = cn \cdot 2^n$.
	Therefore the particular solution become $a_{n}^{(h)} = cn \cdot 2^n = 3n2^n $ So the general solution is the sum the homogeneous solution and this particular solution, namely 
	\begin{eqnarray*}{\fontsize{10}{10pt}}{\arraycolsep=1pt}
	 a_{n} & = & a_{n}^{(h)} + a_{n}^{(p)} \\
	 ~ & = & \alpha \cdot 2^n + 3n2^n \\
	 ~ & = & (3n + \alpha )2^n \\
	\end{eqnarray*}
	\end{frame}		%----------------------------------------------------------------------------------
	\begin{frame}
	\frametitle{Problem 3 (8.3 Exercise 8)}
	\fontsize{10}{10pt}\selectfont
	\begin{enumerate}[label=(\alph*)]
	\item $f~(2) = 2 \cdot 5 + 3 = 13$\\
	\item $f~(4) & = & 2 \cdot 13 + 3 = 29$\\
	$f~(8) & = & 2 \cdot 29 + 3 = 61$\\	
	\item $f~(16) & = & 2 \cdot 61 + 3 = 125$\\
	$f~(32) & = & 2 \cdot 125 + 3 = 253$\\
	$f~(64) & = & 2 \cdot 253 + 3 = 509$\\	
	\item $f~(128) & = & 2 \cdot 509 + 3 = 1021$\\
	$f~(256) & = & 2 \cdot 1021 + 3 = 2045$\\
	$f~(512) & = & 2 \cdot 2045 + 3 = 4093$\\
	$f~(1024) & = & 2 \cdot 4093 + 3 = 8189$\\ 
	\end{enumerate}
	\end{frame}
	
	%----------------------------------------------------------------------------------
	\begin{frame}
	\frametitle{Problem 4 (8.4 Exercise 4(a))}
	\fontsize{10}{10pt}\selectfont
	$ G(x) = a_{0} + a_{1}x + a_{2}x^2 + ... + a_{k}x^k + .... = \mathop{\sum\limits_{k=0}^{+\infty}} a_{k} x^k  $\\
	\vspace{0.3cm}
 Apparently all the terms are 0 except for the seven −1’s shown.\\
	\vspace{0.5cm}
	$ G(x) & = & - 1 - x - x^2 - ... - x^6 + 0x^7 + 0x^8 + ... $\\
	~ & ~ &  = &$ - 1 - x - x^2 - ... - x^6 $\\
	~ & ~ & = &\mathop{\sum\limits_{k=0}^{6}}- x^k \\
	~ & ~ & = &-\mathop{\sum\limits_{k=0}^{6}}x^k \\
	~ & ~ & = $&- \frac{1-x^7}{1-x}$ \\
	\end{frame}
	
    %----------------------------------------------------------------------------------
	\begin{frame}
	\frametitle{Problem 5 (8.5 Exercise 2)}
	\fontsize{10}{10pt}\selectfont
	\vspace{0.5cm}
	$ \left | C\cup D \right | & = & \left | C \right | + \left | D \right | - \left | C\cup D \right |  $ \\
	
	~ & ~ & ~ & ~& =345+212-188\\
	~ & ~ & ~ & ~& =369 \\
	\end{frame}
	
    %----------------------------------------------------------------------------------
	\begin{frame}
	\frametitle{Problem 6 (8.6 Exercise 8)}
	\fontsize{10}{10pt}\selectfont
	$ 5^7 - C(5,1) \cdot (5 - 1)^7 + C(5,2) \cdot (5 - 2)^7 + C(5,3) \cdot (5 - 3)^7 + C(5,4) \cdot (5 - 4)^7 \\= 16800 $
	\end{frame}
		
	%----------------------------------------------------------------------------------

		
%----------------------------------------------------------------------------------
\section{完成作業小時數}
%----------------------------------------------------------------------------------

%----------------------------------------------------------------------------------	
\begin{frame}
\frametitle{完成作業小時數}
\centerline{\fontsize{16}{16pt}\selectfont{\color{blue}完成作業小時數:\color{red}共\underline{   5   }小時\color{blue}(估算,取整數)}}	
\end{frame}
%----------------------------------------------------------------------------------	

%----------------------------------------------------------------------------------
\end{document}
%----------------------------------------------------------------------------------

