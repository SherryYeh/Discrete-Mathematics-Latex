\documentclass[14pt,hyperref={bookmarks=false}]{beamer}

%\usepackage[no-math]{fontspec} % https://www.ptt.cc/bbs/LaTeX/M.1505577244.A.952.html
\usepackage{xeCJK}
\usepackage{amsfonts}
\usepackage{amssymb}
\usepackage{amsmath}
%\usepackage{musixtex}
\usepackage{enumitem}
\usepackage{algorithm,algorithmic}


\renewcommand\qedsymbol{$\blacktriangleleft$}

%\setCJKmainfont{新細明體}
\setCJKmainfont{標楷體}

\usetheme{Malmoe}
\usecolortheme{dolphin}
\usefonttheme[onlymath]{serif}
%\usefonttheme{serif} % professionalfonts
% default, serif, professionalfonts, structurebold, structureitalicserif, structuresmallcapsserif
\useoutertheme{miniframes} %{infolines}
\usepackage{xmpmulti}

\linespread{1.2}

%\newenvironment{num}
%{\leftmargini=6mm\leftmarginii=8mm\begin{itemize}}
%{\end{itemize}}

%\newenvironment{enu}
%{\leftmargini=6mm\leftmarginii=8mm\begin{enumerate}}
%{\end{enumerate}}

%-------------------------------------------------------------
\title{離散數學 107-2}
\subtitle{Homework 06}
\author{姓名: 葉子瑄。學號: 107590041}
\date{Handout: 2019.05.13 (week-13)}

%-------------------------------------------------------------
% page number
% https://tex.stackexchange.com/questions/137022/how-to-insert-page-number-in-beamer-navigation-bars
%-------------------------------------------------------------

\setbeamertemplate{navigation symbols}
{ \insertslidenavigationsymbol 
  \insertframenavigationsymbol   
  \insertsubsectionnavigationsymbol  
  \insertsectionnavigationsymbol
  \insertdocnavigationsymbol  
  \insertbackfindforwardnavigationsymbol 
  \hspace{1em}  
  \usebeamerfont{footline} 
  \insertframenumber
  ~-~
  \inserttotalframenumber
}
\setcounter{page}{1} 
\pagenumbering{arabic} 


%----------------------------------------------------------------------------------
\begin{document}
%----------------------------------------------------------------------------------

\begin{frame}
\titlepage
\end{frame}

\raggedright

\begin{frame}
% \tiny \scriptsize \footnotesize \small \normalsize \large \Large \LARGE \huge \Huge
%\normalsize
\footnotesize
\tableofcontents
\end{frame}
	
%----------------------------------------------------------------------------------
\section{題目}
%----------------------------------------------------------------------------------

    %----------------------------------------------------------------------------------
	\subsection{題目與注意事項}
	%----------------------------------------------------------------------------------
	
	%----------------------------------------------------------------------------------
	\begin{frame}
	\frametitle{Homework 06 題目}
	\fontsize{10pt}{11pt}\selectfont
	\setlength{\baselineskip}{5pt}
	\begin{columns}
	\begin{column}{0.68\textwidth}
	\begin{enumerate}[label=(Prob. \arabic*)]
	\setlength\itemsep{0em}
	\item page 418, chapter 6.1 Exercise 36
	\item page 427, chapter 6.2 Exercise 38
	\item page 435, chapter 6.3 Exercise 20
	\item page 444, chapter 6.4 Exercise 16
	\item page 455, chapter 6.5 Exercise 12
	\item page 461, chapter 6.6 Exercise 6

	\end{enumerate}
	\end{column}
	
%	\begin{column}{0.45\textwidth}
%	\begin{enumerate}[label=(Prob. \arabic*)]
%	\addtocounter{enumi}{21}
%	\setlength\itemsep{0em}
%	\item page xx, chapter xx Exercises xx	
%	\end{enumerate}
%	\end{column}
	
	\end{columns}
	\end{frame}
	
	%----------------------------------------------------------------------------------
	\begin{frame}
	\frametitle{注意事項}
	\fontsize{10}{10pt}\selectfont
	\begin{enumerate}[label=(\alph*)]
	\item 要熟悉 LaTeX 請翻閱 \ \htmladdnormallink{\color{blue}lshort}{https://ctan.org/tex-archive/info/lshort/}。
	\item 記得在最後一頁,回報\selectfont \color{red}{完成作業小時數}(估算,取整數)\selectfont \color{black}{。}
	\item 將檔案夾命名為 \texttt{hw06\_107820xxx},將檔案夾壓縮成 \texttt{hw06\_107820xxx.zip},上傳到\htmladdnormallink{\color{blue}網路學園}{http://elearning.ntut.edu.tw}。
	\item LaTeX 數學符號請查此表: \ \htmladdnormallink{\color{blue}List of LaTeX mathematical symbols}{https://oeis.org/wiki/List_of_LaTeX_mathematical_symbols}。
	\item 作業抄襲,以零分計。作業提供給他人抄襲,以零分計。
	\item 作業遲交一週內成績打五折,作業遲交超過一週以零分計。
	\end{enumerate}
	\end{frame}

%----------------------------------------------------------------------------------
\section{作答區}
%----------------------------------------------------------------------------------

    %----------------------------------------------------------------------------------
	\subsection{解題}
	%----------------------------------------------------------------------------------
	
	%----------------------------------------------------------------------------------
	\begin{frame}
	\frametitle{Problem 01 (6.1 Exercise 36)}
	\fontsize{12}{16pt}\selectfont
	There are 2 possible image,since the image has to be 0 or 1.\\
	 And the domain contains $n$ elements.\\
	 Terefore,there are $2^n$ different functions.\\
	
	\end{frame}
	
	%----------------------------------------------------------------------------------
	\begin{frame}
	\frametitle{Problem 02 (6.2 Exercise 38)}
	\fontsize{12}{16pt}\selectfont
	There are six computers, and each computer connected to at least one of the other five computers, this means the possible connection for each computer is: 1, 2, 3, 4, 5, by using the Pigeon and Pigeonhole Principle, let these 5 possible connections be the Pigeonholes, and let the six computers be the Pigeons, therefore there are must at least two computers have same number of connections.\\
	
	\end{frame}
	
	%----------------------------------------------------------------------------------
	\begin{frame}
	\frametitle{Problem 03 (6.3 Exercise 20)}
	\fontsize{10}{10pt}\selectfont
	% 請作答
	\begin{enumerate}[label=(\alph*)]
	\item $C(10,3)=120$
	\item $C(10,4)+C(10,3)+C(10,2)+C(10,1)+C(10,0)=210+120+45+10+1=386$
	\item $C(10,7)+C(10+8)+C(10,9)+C(10,10)=120+45+10+1=176$
	\item $ C(10,3)+C(10,4)+C(10,5)+C(10,6)+C(10,7)+C(10+8)+C(10,9)+C(10,10)=120+210+252+210+120+45+10+1=968$
	\end{enumerate}
	\end{frame}

	%----------------------------------------------------------------------------------
	\begin{frame}
	\frametitle{Problem 04 (6.4 Exercise 16)}
	\fontsize{10}{10pt}\selectfont	
	Pascal  identity:\begin{pmatrix}
    n+1\\
	k
\end{pmatrix}=\begin{pmatrix}
    n\\
	k-1
\end{pmatrix}+\begin{pmatrix}
    n\\
	k
\end{pmatrix}\\
	\vspace*{0.3cm}
	\begin{columns}
	\begin{column}{0.4\textwidth}	
    \begin{pmatrix}
    11\\
	1
\end{pmatrix} = 1+10=11 \\
	\begin{pmatrix}
	11\\
	2
\end{pmatrix} = 10+45=55 \\
	\begin{pmatrix}
	11\\
	3
\end{pmatrix} = 45+120=165 \\
	\begin{pmatrix}
	11\\
	4
\end{pmatrix} = 120+210+330 \\
	\begin{pmatrix}
	11\\
	5
\end{pmatrix} = 210+252=462 \\
	\end{column}
	\begin{column}{0.4\textwidth}
    \begin{pmatrix}
	11\\
	6
\end{pmatrix} = 252+210=462 \\
	\begin{pmatrix}
	11\\
	7
\end{pmatrix} = 210+120=330 \\
	\begin{pmatrix}
	11\\
	8
\end{pmatrix} = 120+45=165 \\
	\begin{pmatrix}
	11\\
	9
\end{pmatrix} = 45+10=55 \\
	\begin{pmatrix}
	11\\
	10
\end{pmatrix} = 10+1=11 \\
	\end{column}
	\end{columns}
    \vspace*{0.3cm}
	The row of \begin{pmatrix}
	11\\
	k
\end{pmatrix}is 1 11 55 165 330 462 462 330 165 55 11 1\\
	
	\end{frame}
	
    %----------------------------------------------------------------------------------
	\begin{frame}
	\frametitle{Problem 05 (6.5 Exercise 12)}
	\fontsize{10}{10pt}\selectfont
	 pennies+nickels+dimes+quarters+ half dollars=20\\
	 Therefore by Theorem 2 the answer is C(4+20, 20) = C(24, 20)  = 10626.\\

	\end{frame}
	
    %----------------------------------------------------------------------------------
	\begin{frame}
	\frametitle{Problem 06 (6.6 Exercise 6)}
	\fontsize{10}{10pt}\selectfont
	% 請作答\\
	\begin{enumerate}[label=(\alph*)]
	\item 1423
	\item 51234
	\item 13254
	\item 612354
	\item 1623574
	\item 23587461
	\end{enumerate}
	\end{frame}

	%----------------------------------------------------------------------------------

		
%----------------------------------------------------------------------------------
\section{完成作業小時數}
%----------------------------------------------------------------------------------

%----------------------------------------------------------------------------------	
\begin{frame}
\frametitle{完成作業小時數}
\centerline{\fontsize{16}{16pt}\selectfont{\color{blue}完成作業小時數:\color{red}共\underline{   5   }小時\color{blue}(估算,取整數)}}	
\end{frame}
%----------------------------------------------------------------------------------	

%----------------------------------------------------------------------------------
\end{document}
%----------------------------------------------------------------------------------

