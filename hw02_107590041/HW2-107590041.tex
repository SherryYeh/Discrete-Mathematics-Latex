\documentclass[14pt,hyperref={bookmarks=false}]{beamer}

%\usepackage[no-math]{fontspec} % https://www.ptt.cc/bbs/LaTeX/M.1505577244.A.952.html
\usepackage{xeCJK}
\usepackage{amsfonts}
\usepackage{amssymb}
\usepackage{amsmath}
%\usepackage{musixtex}
\usepackage{enumitem}

\renewcommand\qedsymbol{$\blacktriangleleft$}

%\setCJKmainfont{新細明體}
\setCJKmainfont{標楷體}

\usetheme{Malmoe}
\usecolortheme{dolphin}
\usefonttheme[onlymath]{serif}
%\usefonttheme{serif} % professionalfonts
% default, serif, professionalfonts, structurebold, structureitalicserif, structuresmallcapsserif
\useoutertheme{miniframes} %{infolines}
\usepackage{xmpmulti}

\linespread{1.2}

%\newenvironment{num}
%{\leftmargini=6mm\leftmarginii=8mm\begin{itemize}}
%{\end{itemize}}

%\newenvironment{enu}
%{\leftmargini=6mm\leftmarginii=8mm\begin{enumerate}}
%{\end{enumerate}}

%-------------------------------------------------------------
\title{離散數學 107-2}
\subtitle{Homework 02}
\author{姓名: 葉子瑄。學號: 107590041}
\date{截止收件: 2019.04.10 (Wednesday) 23:59 pm \\ (week-8)}

%-------------------------------------------------------------
% page number
% https://tex.stackexchange.com/questions/137022/how-to-insert-page-number-in-beamer-navigation-bars
%-------------------------------------------------------------

\setbeamertemplate{navigation symbols}
{ \insertslidenavigationsymbol 
  \insertframenavigationsymbol   
  \insertsubsectionnavigationsymbol  
  \insertsectionnavigationsymbol
  \insertdocnavigationsymbol  
  \insertbackfindforwardnavigationsymbol 
  \hspace{1em}  
  \usebeamerfont{footline} 
  \insertframenumber
  ~-~
  \inserttotalframenumber
}
\setcounter{page}{1} 
\pagenumbering{arabic} 


%----------------------------------------------------------------------------------
\begin{document}
%----------------------------------------------------------------------------------

\begin{frame}
\titlepage
\end{frame}

\raggedright

\begin{frame}
% \tiny \scriptsize \footnotesize \small \normalsize \large \Large \LARGE \huge \Huge
%\normalsize
\footnotesize
\tableofcontents
\end{frame}
	
%----------------------------------------------------------------------------------
\section{題目}
%----------------------------------------------------------------------------------

    %----------------------------------------------------------------------------------
	\subsection{題目與注意事項}
	%----------------------------------------------------------------------------------
	
	%----------------------------------------------------------------------------------
	\begin{frame}
	\frametitle{Homework 02 題目}
	\fontsize{8pt}{9pt}\selectfont
	\setlength{\baselineskip}{5pt}
	\begin{columns}
	\begin{column}{0.68\textwidth}
	\begin{enumerate}[label=(Prob. \arabic*)]
	\setlength\itemsep{0em}
	\item page 132, chapter 2.1 Exercises 36
	\item page 144, chapter 2.2 Exercises 4
	\item page 163, chapter 2.3 Exercises 38
	\item page 179, chapter 2.4 Exercises 34
	\item page 186, chapter 2.5 Exercises 6
	\item page 194, chapter 2.6 Exercises 4
	\end{enumerate}
	\end{column}
	
%	\begin{column}{0.45\textwidth}
%	\begin{enumerate}[label=(Prob. \arabic*)]
%	\addtocounter{enumi}{21}
%	\setlength\itemsep{0em}
%	\item page xx, chapter xx Exercises xx	
%	\end{enumerate}
%	\end{column}
	
	\end{columns}
	\end{frame}
	
	%----------------------------------------------------------------------------------
	\begin{frame}
	\frametitle{注意事項}
	\fontsize{10}{10pt}\selectfont
	\begin{enumerate}[label=(\alph*)]
	\item 要熟悉 LaTeX 請翻閱 \ \htmladdnormallink{\color{blue}lshort}{https://ctan.org/tex-archive/info/lshort/}。
	\item 記得在最後一頁,回報\selectfont \color{red}{完成作業小時數}(估算,取整數)\selectfont \color{black}{。}
	\item 將檔案夾命名為 \texttt{hw02\_107820xxx},將檔案夾壓縮成 \texttt{hw02\_107820xxx.zip},上傳到\htmladdnormallink{\color{blue}網路學園}{http://elearning.ntut.edu.tw}。
	\item LaTeX 數學符號請查此表: \ \htmladdnormallink{\color{blue}List of LaTeX mathematical symbols}{https://oeis.org/wiki/List_of_LaTeX_mathematical_symbols}。
	\item 作業抄襲,以零分計。作業提供給他人抄襲,以零分計。
	\item 作業遲交一週內成績打五折,作業遲交超過一週以零分計。
	\end{enumerate}
	\end{frame}

%----------------------------------------------------------------------------------
\section{作答區}
%----------------------------------------------------------------------------------

    %----------------------------------------------------------------------------------
	\subsection{解題}
	%----------------------------------------------------------------------------------
	
	
	%----------------------------------------------------------------------------------
	\begin{frame}
	\frametitle{Problem 1 (2.1 Exercises 36)}
	\fontsize{9.5}{10pt}\selectfont
	\begin{enumerate}[label=(\alph*)]
	\item $\{(a, a, a)\}$
	\item $\{(0,0,0),(0,0,a),(0,a,0),(0,a,a),(a,0,0),(a,0,a),(a,a,0),(a,a,a)\}$
	\end{enumerate}
	\end{frame}
	
	%----------------------------------------------------------------------------------
	\begin{frame}
	\frametitle{Problem 2 (2.2 Exercises 4)}
	\fontsize{10}{10pt}\selectfont
	\begin{enumerate}[label=(\alph*)]
	\item $\{a, b, c, d, e, f, g, h\}$
	\item $\{a,b,c,d,e\}$
	\item $~\{\}$
	\item $\{f,g,h\}$
	\end{enumerate}
	\end{frame}

	%----------------------------------------------------------------------------------
	\begin{frame}
	\frametitle{Problem 3 (2.3 Exercises 38)}
	\fontsize{10}{10pt}\selectfont
	答:\\
	Given:~~$g$:$R \rightarrow R$~and~$f$:$R \rightarrow R$\\
	$~~~~~~~~~~f(x)=x^2+1$\\
	$~~~~~~~~~~g(x)=x+2$\\
	$\therefore f \circ g~and~g \circ f$ are also from $R$ to $R$\\
	$(f \circ g)(x)=f(g(x))=f(x+2)=(x+2)^2+1=x^2+4x+5$\\
	$(g \circ f)(x)=g(f(x)=g(x^2+1)=(x^2+1)+2=x^2+3$\\
	\end{frame}

	%----------------------------------------------------------------------------------
	\begin{frame}
	\frametitle{Problem 4 (2.4 Exercises 34)}
	\fontsize{10}{10pt}\selectfont
	\begin{enumerate}[label=(\alph*)]
	\item $(1−1)+(1-2)+(2−1)+(2−2)+(3−1)+(3−2)=3$
	\item $0+2+4+3+5+7+6+8+10+9+11+13=78
	 $
	\item $0+1+2+0+1+2+0+1+2=9$
	\item $0+0+1+8+27+4+32+108=180$
	\end{enumerate}
	\end{frame}
	
	%----------------------------------------------------------------------------------
	\begin{frame}
	\frametitle{Problem 5 (2.5 Exercises 6)}
	\fontsize{10}{10pt}\selectfont
	答:\\
	$\because$ Hotel has a countable infinite number of rooms,we can number the  positive integers $\mathbf{Z^+}:1,2,3,...$\\
	Given~:the hotel closes all the even numbered rooms\\
	$f:\mathbf{Z^+} \rightarrow \mathbf{Z^+},f(n)=2n-1$\\
	$\therefore$ Move each guest from room $n$ to room $2n-1$
	\end{frame}
	
	%----------------------------------------------------------------------------------
	\begin{frame}
	\frametitle{Problem 6 (2.6 Exercises 4)}
	\fontsize{8}{9pt}\selectfont

	(a)
	\begin{equation*}
	\begin{bmatrix}
	    1 &  0 &  1 \\
 	    0 & -1 & -1 \\
 	   -1 &  1 &  0
	\end{bmatrix}
	\begin{bmatrix}
	    0 &  1 & -1 \\
 	    1 & -1 &  0 \\
 	   -1 &  0 &  1
	\end{bmatrix}	
	=
	\begin{bmatrix}
	   -1 &  1 &  0 \\
 	    0 &  1 & -1 \\
 	    1 & -2 &  1
	\end{bmatrix}
	\end{equation*}
	
	(b)
	\begin{equation*}
	\begin{bmatrix}
		1 & -3 & 0 \\
		1 & 2  & 2 \\
		2 & 1  & -1 \\
	\end{bmatrix}
	\begin{bmatrix}
		1 & -1 & 2 & 3\\
		-1 & 0 & 3 & -1\\
		-3 & -2 & 0 & 2
	\end{bmatrix}
	=
	\begin{bmatrix}
		4 & -1 & -7 & 6\\
		-7 & -5 & 8 & 5\\
		4 & 0 & 7 & 3
	\end{bmatrix}
	\end{equation*}

	(c)
	\begin{equation*}
	\begin{bmatrix}
		0 & -1 \\
		7 & 2\\
		-4 & -3
	\end{bmatrix}
	\begin{bmatrix}
	 4 & -1 & 2 & 3 & 0\\
	 -2 & 0 & 3 & 4 & 1
	\end{bmatrix}
	=
	\begin{bmatrix}
		2 & 0 & -3 & -4 & -1\\
		24 & -7 & 20 & 29 & 2\\
		-10 & 4 & -17 & -24 & -3
	\end{bmatrix}
	\end{equation*}
	
	\end{frame}					
	%----------------------------------------------------------------------------------

	
	%----------------------------------------------------------------------------------

		
%----------------------------------------------------------------------------------
\section{完成作業小時數}
%----------------------------------------------------------------------------------

%----------------------------------------------------------------------------------	
\begin{frame}
\frametitle{完成作業小時數}
\centerline{\fontsize{16}{16pt}\selectfont{\color{blue}完成作業小時數:\color{red}共\underline{   3   }小時\color{blue}(估算,取整數)}}	
\end{frame}
%----------------------------------------------------------------------------------	

%----------------------------------------------------------------------------------
\end{document}
%----------------------------------------------------------------------------------

