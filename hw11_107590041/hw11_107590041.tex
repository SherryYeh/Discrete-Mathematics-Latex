\documentclass[12pt,hyperref={bookmarks=false}]{beamer}

%\usepackage[no-math]{fontspec} % https://www.ptt.cc/bbs/LaTeX/M.1505577244.A.952.html
\usepackage{xeCJK}
\usepackage{amsfonts}
\usepackage{amssymb}
\usepackage{amsmath}
%\usepackage{musixtex}
\usepackage{enumitem}
\usepackage{algorithm,algorithmic}


\renewcommand\qedsymbol{$\blacktriangleleft$}

%\setCJKmainfont{新細明體}
\setCJKmainfont{標楷體}

\usetheme{Malmoe}
\usecolortheme{dolphin}
\usefonttheme[onlymath]{serif}
%\usefonttheme{serif} % professionalfonts
% default, serif, professionalfonts, structurebold, structureitalicserif, structuresmallcapsserif
\useoutertheme{miniframes} %{infolines}
\usepackage{xmpmulti}

\linespread{1.2}

%\newenvironment{num}
%{\leftmargini=6mm\leftmarginii=8mm\begin{itemize}}
%{\end{itemize}}

%\newenvironment{enu}
%{\leftmargini=6mm\leftmarginii=8mm\begin{enumerate}}
%{\end{enumerate}}

%-------------------------------------------------------------
\title{離散數學 107-2}
\subtitle{Homework 11}
\author{姓名: 葉子瑄。學號: 107590041}
\date{Handout: 2019.06.03 (week-16)}

%-------------------------------------------------------------
% page number
% https://tex.stackexchange.com/questions/137022/how-to-insert-page-number-in-beamer-navigation-bars
%-------------------------------------------------------------

\setbeamertemplate{navigation symbols}
{ \insertslidenavigationsymbol 
  \insertframenavigationsymbol   
  \insertsubsectionnavigationsymbol  
  \insertsectionnavigationsymbol
  \insertdocnavigationsymbol  
  \insertbackfindforwardnavigationsymbol 
  \hspace{1em}  
  \usebeamerfont{footline} 
  \insertframenumber
  ~-~
  \inserttotalframenumber
}
\setcounter{page}{1} 
\pagenumbering{arabic} 


%----------------------------------------------------------------------------------
\begin{document}
%----------------------------------------------------------------------------------

\begin{frame}
\titlepage
\end{frame}

\raggedright

\begin{frame}
% \tiny \scriptsize \footnotesize \small \normalsize \large \Large \LARGE \huge \Huge
%\normalsize
\footnotesize
\tableofcontents
\end{frame}
	
%----------------------------------------------------------------------------------
\section{題目}
%----------------------------------------------------------------------------------

    %----------------------------------------------------------------------------------
	\subsection{題目與注意事項}
	%----------------------------------------------------------------------------------
	
	%----------------------------------------------------------------------------------
	\begin{frame}
	\frametitle{Homework 11 題目}
	\fontsize{10pt}{11pt}\selectfont
	\setlength{\baselineskip}{5pt}
	\begin{columns}
	\begin{column}{0.68\textwidth}
	\begin{enumerate}[label=(Prob. \arabic*)]
	\setlength\itemsep{0em}
	\item page 791, chapter 11.1 Exercise 4
	\item page 806, chapter 11.2 Exercise 22
	\item page 820, chapter 11.3 Exercise 8
	\item page 832, chapter 11.4 Exercise 12
	\item page 839, chapter 11.5 Exercise 2
	\end{enumerate}
	\end{column}
	
%	\begin{column}{0.45\textwidth}
%	\begin{enumerate}[label=(Prob. \arabic*)]
%	\addtocounter{enumi}{21}
%	\setlength\itemsep{0em}
%	\item page xx, chapter xx Exercises xx	
%	\end{enumerate}
%	\end{column}
	
	\end{columns}
	\end{frame}
	
	%----------------------------------------------------------------------------------
	\begin{frame}
	\frametitle{注意事項}
	\fontsize{10}{10pt}\selectfont
	\begin{enumerate}[label=(\alph*)]
	\item 要熟悉 LaTeX 請翻閱 \ \htmladdnormallink{\color{blue}lshort}{https://ctan.org/tex-archive/info/lshort/}。
	\item 記得在最後一頁,回報\selectfont \color{red}{完成作業小時數}(估算,取整數)\selectfont \color{black}{。}
	\item 將檔案夾命名為 \texttt{hw11\_107820xxx},將檔案夾壓縮成 \texttt{hw11\_107820xxx.zip},上傳到\htmladdnormallink{\color{blue}網路學園}{http://elearning.ntut.edu.tw}。
	\item LaTeX 數學符號請查此表: \ \htmladdnormallink{\color{blue}List of LaTeX mathematical symbols}{https://oeis.org/wiki/List_of_LaTeX_mathematical_symbols}。
	\item 作業抄襲,以零分計。作業提供給他人抄襲,以零分計。
	\item 作業遲交一週內成績打五折,作業遲交超過一週以零分計。
	\end{enumerate}
	\end{frame}

%----------------------------------------------------------------------------------
\section{作答區}
%----------------------------------------------------------------------------------

    %----------------------------------------------------------------------------------
	\subsection{解題}
	%----------------------------------------------------------------------------------

	%----------------------------------------------------------------------------------
	\begin{frame}
	\frametitle{Problem 1 (11.1 Exercise 4)}
	\fontsize{12}{16pt}\selectfont
	\begin{enumerate}[label=(\alph*)]
	\setlength\itemsep{0em}
	\item vertex $a$ is the root
	\item vertex $a$, $b$, $d$, $e$, $g$, $h$, $i$, and $o$ are internal vertices.
	\item vertex $c$, $f$, $j$, $k$, $l$, $m$, $n$, $p$, $q$, $r$, and $s$ are leaves vertices.
	\item vertex $j$ does not have any children.
	\item The parent of $h$ is the vertex $d$.
	\item The siblings of $o$ is the vertex $p$.
	\item The ancestors of $m$ are $a$ , $b$, and $g$.
	\item The descendants of $b$ are $e$, $f$ , $g$ , $j$ , $k$ , $l$, and $m$.
	\end{enumerate}
	\end{frame}

	%----------------------------------------------------------------------------------
	\begin{frame}
	\frametitle{Problem 2 (11.2 Exercise 22)}
	\fontsize{12}{16pt}\selectfont
	\begin{enumerate}[label=(\alph*)]
	\setlength\itemsep{0em}
	\item test
	\item beer
	\item sex
	\item tax
	\end{enumerate}
	\end{frame}
	

	%----------------------------------------------------------------------------------
	\begin{frame}
	\frametitle{Problem 3 (11.3 Exercise 8)}
	\fontsize{12}{16pt}\selectfont
	Ans:$a$, $b$, $d$, $e$, $i$, $j$, $m$, $n$, $o$, $c$, $f$, $g$, $h$, $k$, $l$, $p$.
	\end{frame}

	%----------------------------------------------------------------------------------
	\begin{frame}
	\frametitle{Problem 4 (11.4 Exercise 12)}
	\fontsize{10}{10pt}\selectfont
	\begin{enumerate}[label=(\alph*)]
	\setlength\itemsep{0em}
	\item 1
	\item 2
	\item 3
	\end{enumerate}
	\end{frame}
	
	%----------------------------------------------------------------------------------
	\begin{frame}
	\frametitle{Problem 5 (11.5 Exercise 2)}
	\fontsize{10}{10pt}\selectfont

	\vspace*{0.3cm}
	Ans: \\
	We start with the minimum weight edge ${a, b}$. The least weight edge incident to the tree constructed so far is edge $\{a, e\}$, with weight 2, so we add it to the tree. Next we add edge $\{c, e\}$, and then edge $\{c, d\}$. This completes the tree, whose total weight is \underline{~~6~~}.
	\end{frame}
	
	%----------------------------------------------------------------------------------


		
%----------------------------------------------------------------------------------
\section{完成作業小時數}
%----------------------------------------------------------------------------------

%----------------------------------------------------------------------------------	
\begin{frame}
\frametitle{完成作業小時數}
\centerline{\fontsize{16}{16pt}\selectfont{\color{blue}完成作業小時數:\color{red}共\underline{   3   }小時\color{blue}(估算,取整數)}}	
\end{frame}
%----------------------------------------------------------------------------------	

%----------------------------------------------------------------------------------
\end{document}
%----------------------------------------------------------------------------------

