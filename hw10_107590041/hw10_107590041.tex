\documentclass[12pt,hyperref={bookmarks=false}]{beamer}

%\usepackage[no-math]{fontspec} % https://www.ptt.cc/bbs/LaTeX/M.1505577244.A.952.html
\usepackage{xeCJK}
\usepackage{amsfonts}
\usepackage{amssymb}
\usepackage{amsmath}
%\usepackage{musixtex}
\usepackage{enumitem}
\usepackage{algorithm,algorithmic}


\renewcommand\qedsymbol{$\blacktriangleleft$}

%\setCJKmainfont{新細明體}
\setCJKmainfont{標楷體}

\usetheme{Malmoe}
\usecolortheme{dolphin}
\usefonttheme[onlymath]{serif}
%\usefonttheme{serif} % professionalfonts
% default, serif, professionalfonts, structurebold, structureitalicserif, structuresmallcapsserif
\useoutertheme{miniframes} %{infolines}
\usepackage{xmpmulti}

\linespread{1.2}

%\newenvironment{num}
%{\leftmargini=6mm\leftmarginii=8mm\begin{itemize}}
%{\end{itemize}}

%\newenvironment{enu}
%{\leftmargini=6mm\leftmarginii=8mm\begin{enumerate}}
%{\end{enumerate}}

%-------------------------------------------------------------
\title{離散數學 107-2}
\subtitle{Homework 10}
\author{姓名: 葉子瑄。學號: 107590041}
\date{Handout: 2019.06.3 (week-16)}

%-------------------------------------------------------------
% page number
% https://tex.stackexchange.com/questions/137022/how-to-insert-page-number-in-beamer-navigation-bars
%-------------------------------------------------------------

\setbeamertemplate{navigation symbols}
{ \insertslidenavigationsymbol 
  \insertframenavigationsymbol   
  \insertsubsectionnavigationsymbol  
  \insertsectionnavigationsymbol
  \insertdocnavigationsymbol  
  \insertbackfindforwardnavigationsymbol 
  \hspace{1em}  
  \usebeamerfont{footline} 
  \insertframenumber
  ~-~
  \inserttotalframenumber
}
\setcounter{page}{1} 
\pagenumbering{arabic} 


%----------------------------------------------------------------------------------
\begin{document}
%----------------------------------------------------------------------------------

\begin{frame}
\titlepage
\end{frame}

\raggedright

\begin{frame}
% \tiny \scriptsize \footnotesize \small \normalsize \large \Large \LARGE \huge \Huge
%\normalsize
\footnotesize
\tableofcontents
\end{frame}
	
%----------------------------------------------------------------------------------
\section{題目}
%----------------------------------------------------------------------------------

    %----------------------------------------------------------------------------------
	\subsection{題目與注意事項}
	%----------------------------------------------------------------------------------
	
	%----------------------------------------------------------------------------------
	\begin{frame}
	\frametitle{Homework 10 題目}
	\fontsize{10pt}{11pt}\selectfont
	\setlength{\baselineskip}{5pt}
	\begin{columns}
	\begin{column}{0.68\textwidth}
	\begin{enumerate}[label=(Prob. \arabic*)]
	\setlength\itemsep{0em}
	\item page 683, chapter 10.1 Exercise 2
	\item page 699, chapter 10.2 Exercise 8
	\item page 710, chapter 10.3 Exercise 8
	\item page 724, chapter 10.4 Exercise 2
	\item page 741, chapter 10.5 Exercise 54
	\item page 753, chapter 10.6 Exercise 26
	\item page 761, chapter 10.7 Exercise 14
	\item page 768, chapter 10.8 Exercise 6
	\end{enumerate}
	\end{column}
	
%	\begin{column}{0.45\textwidth}
%	\begin{enumerate}[label=(Prob. \arabic*)]
%	\addtocounter{enumi}{21}
%	\setlength\itemsep{0em}
%	\item page xx, chapter xx Exercises xx	
%	\end{enumerate}
%	\end{column}
	
	\end{columns}
	\end{frame}
	
	%----------------------------------------------------------------------------------
	\begin{frame}
	\frametitle{注意事項}
	\fontsize{10}{10pt}\selectfont
	\begin{enumerate}[label=(\alph*)]
	\item 要熟悉 LaTeX 請翻閱 \ \htmladdnormallink{\color{blue}lshort}{https://ctan.org/tex-archive/info/lshort/}。
	\item 記得在最後一頁,回報\selectfont \color{red}{完成作業小時數}(估算,取整數)\selectfont \color{black}{。}
	\item 將檔案夾命名為 \texttt{hw10\_107820xxx},將檔案夾壓縮成 \texttt{hw10\_107820xxx.zip},上傳到\htmladdnormallink{\color{blue}網路學園}{http://elearning.ntut.edu.tw}。
	\item LaTeX 數學符號請查此表: \ \htmladdnormallink{\color{blue}List of LaTeX mathematical symbols}{https://oeis.org/wiki/List_of_LaTeX_mathematical_symbols}。
	\item 作業抄襲,以零分計。作業提供給他人抄襲,以零分計。
	\item 作業遲交一週內成績打五折,作業遲交超過一週以零分計。
	\end{enumerate}
	\end{frame}

%----------------------------------------------------------------------------------
\section{作答區}
%----------------------------------------------------------------------------------

    %----------------------------------------------------------------------------------
	\subsection{解題}
	%----------------------------------------------------------------------------------

	%----------------------------------------------------------------------------------
	\begin{frame}
	\frametitle{Problem 1 (10.1 Exercise 2)}
	\fontsize{12}{16pt}\selectfont
	\begin{enumerate}[label=(\alph*)]
	\setlength\itemsep{0em}
	\item simple graph
	\item multigraph
	\item pseudograph
	\end{enumerate}
	\end{frame}

	%----------------------------------------------------------------------------------
	\begin{frame}
	\frametitle{Problem 2 (10.2 Exercise 8)}
	\fontsize{12}{16pt}\selectfont
	
	Ans: In this directed multigraph there are \underline{~~4~~} vertices and \underline{~~8~~} edges.\\ The degrees are:\\ 
	$\text{deg}^{-}(a) = 2$, $\text{deg}^{+}(a) = 2$, \\
	$\text{deg}^{-}(b) = 3$, $\text{deg}^{+}(b) = 4$, \\
	$\text{deg}^{-}(c) = 2$, $\text{deg}^{+}(c) = 1$, \\
	$\text{deg}^{-}(d) = 1$, $\text{deg}^{+}(d) = 1$, \\	
	\end{frame}
	

	%----------------------------------------------------------------------------------
	\begin{frame}
	\frametitle{Problem 3 (10.3 Exercise 8)}
	\fontsize{12}{16pt}\selectfont

	$
	\begin{bmatrix}
	   0 & 1 & 0 & 1 & 0 \\
 	   1 & 0 & 1 & 1 & 1 \\
 	   0 & 1 & 1 & 0 & 0 \\
 	   1 & 0 & 0 & 0 & 1 \\
 	   0 & 0 & 1 & 0 & 1 \\
	\end{bmatrix}
	$

	\end{frame}

	%----------------------------------------------------------------------------------
	\begin{frame}
	\frametitle{Problem 4 (10.4 Exercise 2)}
	\fontsize{10}{10pt}\selectfont
	\begin{enumerate}[label=(\alph*)]
	\setlength\itemsep{1em}
	\item This \underline{~~is~~} (is, is not) a path of length \underline{~~4~~}, it \underline{~~is not~~} (is, is not) a circuit, it \underline{~~is~~} (is, is not) simple.

	\item This \underline{~~is~~} (is, is not) a path of length \underline{~~4~~}, it \underline{~~is~~} (is, is not) a circuit, it \underline{~~is not~~} (is, is not) simple.

	\item This \underline{~~is not~~} (is, is not) a path, since there is no edge from d to b.
	
	\item This \underline{~~is not~~} (is, is not) a path, since there is no edge from b to d.
	\end{enumerate}
	\end{frame}
	
	%----------------------------------------------------------------------------------
	\begin{frame}
	\frametitle{Problem 5 (10.5 Exercise 54)}
	\fontsize{10}{10pt}\selectfont

	\vspace*{0.3cm}
	Ans: \\
	\vspace*{0.3cm}

	\underline{~~. An Euler~~} (An Euler, A Hamilton) path will cover every link, so it can be used to test the links.\\

	\vspace*{0.3cm}

	\underline{~~A Hamilton~~} (An Euler, A Hamilton) path will cover all the devices, so it can be used to test the devices.
	
	\end{frame}
	
    %----------------------------------------------------------------------------------
	\begin{frame}
	\frametitle{Problem 6 (10.6 Exercise 26)}
	\fontsize{9}{9pt}\selectfont

	\begin{enumerate}[label=(\alph*)]
	\setlength\itemsep{0.2em}
	\item Circuit: a-b-c-d-e-a, Weight = 3 + 10 + 6 + 1 + 7 = 27
	\item Circuit: a-b-c-e-d-a, Weight = 3 + 10 + 5 + 1 + 4 = 23
	\item Circuit: a-b-d-c-e-a, Weight = 3 + 9 + 6 + 5 + 7 = 30
	\item Circuit: a-b-d-e-c-a, Weight = 3 + 9 + 1 + 5 + 8 = 26

	
	\item Circuit: a-b-e-c-d-a, Weight = 3 + 2 + 5 + 6 + 4 = 20
	\item Circuit: a-b-e-d-c-a, Weight = 3  + 2 + 1 + 6 + 8 = 20
	\item Circuit: a-c-b-d-e-a, Weight = 8 + 10 + 9 + 1 + 7 = 35
	\item Circuit: a-c-b-e-d-a, Weight = 8 + 10 + 2 + 1 + 4 = 25
	
	\item Circuit: a-c-d-b-e-a, Weight = 8 + 6 + 9 + 2 + 7 = 32
	\item Circuit: a-c-e-b-d-a, Weight = 8 + 5 + 2 + 9 + 4 = 28
	\item Circuit: a-d-b-c-e-a, Weight = 4 + 9 + 10 + 5 + 7 = 35
	\item Circuit: a-d-c-b-e-a, Weight = 4 + 6 + 10 + 2 + 7 = 29
	\end{enumerate}
	The circuits \underline{~~a,b,e,c,d,a~~} and \underline{~~a,b,e,d,c,a~~} are the ones with minimum total weight.
	\end{frame}


	%----------------------------------------------------------------------------------
	\begin{frame}
	\frametitle{Problem 7 (10.7 Exercise 14)}
	\fontsize{12}{16pt}\selectfont

	\vspace*{0.3cm}
	Ans: \\
	\vspace*{0.3cm}
	Euler's formula says that $v - e + r = 2$.\\
	We are given $e = 30$ and $r = 20$. \\
	Therefore $v = e - r +2  = 30 − 20 + 2 = 12$.
	\end{frame}

	%----------------------------------------------------------------------------------
	\begin{frame}
	\frametitle{Problem 8 (10.8 Exercise 6)}
	\fontsize{12}{16pt}\selectfont
	Ans: Since there is a triangle in the graph, we will need at least 3 colors.
	Because vertex g is connected to all other vertices,so it needs to be assigned a unique color. 
	Then we can assign one color to b,d and f and another to a,c and e. 
	\end{frame}	
	%----------------------------------------------------------------------------------


		
%----------------------------------------------------------------------------------
\section{完成作業小時數}
%----------------------------------------------------------------------------------

%----------------------------------------------------------------------------------	
\begin{frame}
\frametitle{完成作業小時數}
\centerline{\fontsize{16}{16pt}\selectfont{\color{blue}完成作業小時數:\color{red}共\underline{   4   }小時\color{blue}(估算,取整數)}}	
\end{frame}
%----------------------------------------------------------------------------------	

%----------------------------------------------------------------------------------
\end{document}
%----------------------------------------------------------------------------------

