\documentclass[12pt,hyperref={bookmarks=false}]{beamer}

%\usepackage[no-math]{fontspec} % https://www.ptt.cc/bbs/LaTeX/M.1505577244.A.952.html
\usepackage{xeCJK}
\usepackage{amsfonts}
\usepackage{amssymb}
\usepackage{amsmath}
%\usepackage{musixtex}
\usepackage{enumitem}
\usepackage{algorithm,algorithmic}


\renewcommand\qedsymbol{$\blacktriangleleft$}

%\setCJKmainfont{新細明體}
\setCJKmainfont{標楷體}

\usetheme{Malmoe}
\usecolortheme{dolphin}
\usefonttheme[onlymath]{serif}
%\usefonttheme{serif} % professionalfonts
% default, serif, professionalfonts, structurebold, structureitalicserif, structuresmallcapsserif
\useoutertheme{miniframes} %{infolines}
\usepackage{xmpmulti}

\linespread{1.2}

%\newenvironment{num}
%{\leftmargini=6mm\leftmarginii=8mm\begin{itemize}}
%{\end{itemize}}

%\newenvironment{enu}
%{\leftmargini=6mm\leftmarginii=8mm\begin{enumerate}}
%{\end{enumerate}}

%-------------------------------------------------------------
\title{離散數學 107-2}
\subtitle{Homework 12}
\author{姓名: 葉子瑄。學號: 107590041}
\date{Handout: 2019.06.10 (week-17)}

%-------------------------------------------------------------
% page number
% https://tex.stackexchange.com/questions/137022/how-to-insert-page-number-in-beamer-navigation-bars
%-------------------------------------------------------------

\setbeamertemplate{navigation symbols}
{ \insertslidenavigationsymbol 
  \insertframenavigationsymbol   
  \insertsubsectionnavigationsymbol  
  \insertsectionnavigationsymbol
  \insertdocnavigationsymbol  
  \insertbackfindforwardnavigationsymbol 
  \hspace{1em}  
  \usebeamerfont{footline} 
  \insertframenumber
  ~-~
  \inserttotalframenumber
}
\setcounter{page}{1} 
\pagenumbering{arabic} 


%----------------------------------------------------------------------------------
\begin{document}
%----------------------------------------------------------------------------------

\begin{frame}
\titlepage
\end{frame}

\raggedright

\begin{frame}
% \tiny \scriptsize \footnotesize \small \normalsize \large \Large \LARGE \huge \Huge
%\normalsize
\footnotesize
\tableofcontents
\end{frame}
	
%----------------------------------------------------------------------------------
\section{題目}
%----------------------------------------------------------------------------------

    %----------------------------------------------------------------------------------
	\subsection{題目與注意事項}
	%----------------------------------------------------------------------------------
	
	%----------------------------------------------------------------------------------
	\begin{frame}
	\frametitle{Homework 12 題目}
	\fontsize{10pt}{11pt}\selectfont
	\setlength{\baselineskip}{5pt}
	\begin{columns}
	\begin{column}{0.68\textwidth}
	\begin{enumerate}[label=(Prob. \arabic*)]
	\setlength\itemsep{0em}
	\item page 854, chapter 12.1 Exercise 2
	\item page 858, chapter 12.2 Exercise 2
	\item page 863, chapter 12.3 Exercise 4
	\item page 877, chapter 12.4 Exercise 2
	\end{enumerate}
	\end{column}
	
%	\begin{column}{0.45\textwidth}
%	\begin{enumerate}[label=(Prob. \arabic*)]
%	\addtocounter{enumi}{21}
%	\setlength\itemsep{0em}
%	\item page xx, chapter xx Exercises xx	
%	\end{enumerate}
%	\end{column}
	
	\end{columns}
	\end{frame}
	
	%----------------------------------------------------------------------------------
	\begin{frame}
	\frametitle{注意事項}
	\fontsize{10}{10pt}\selectfont
	\begin{enumerate}[label=(\alph*)]
	\item 要熟悉 LaTeX 請翻閱 \ \htmladdnormallink{\color{blue}lshort}{https://ctan.org/tex-archive/info/lshort/}。
	\item 記得在最後一頁,回報\selectfont \color{red}{完成作業小時數}(估算,取整數)\selectfont \color{black}{。}
	\item 將檔案夾命名為 \texttt{hw12\_107820xxx},將檔案夾壓縮成 \texttt{hw12\_107820xxx.zip},上傳到\htmladdnormallink{\color{blue}網路學園}{http://elearning.ntut.edu.tw}。
	\item LaTeX 數學符號請查此表: \ \htmladdnormallink{\color{blue}List of LaTeX mathematical symbols}{https://oeis.org/wiki/List_of_LaTeX_mathematical_symbols}。
	\item 作業抄襲,以零分計。作業提供給他人抄襲,以零分計。
	\item 作業遲交一週內成績打五折,作業遲交超過一週以零分計。
	\end{enumerate}
	\end{frame}

%----------------------------------------------------------------------------------
\section{作答區}
%----------------------------------------------------------------------------------

    %----------------------------------------------------------------------------------
	\subsection{解題}
	%----------------------------------------------------------------------------------

	%----------------------------------------------------------------------------------
	\begin{frame}
	\frametitle{Problem 1 (12.1 Exercise 2)}
	\fontsize{12}{16pt}\selectfont
	\begin{enumerate}[label=(\alph*)]
	\setlength\itemsep{0em}
	\item $x = 0$
	\item $x=0$
	\item $x=0$ and $x=1$
	\item No Values
	\end{enumerate}
	\end{frame}

	%----------------------------------------------------------------------------------
	\begin{frame}
	\frametitle{Problem 2 (12.2 Exercise 2)}
	\fontsize{10}{12pt}\selectfont
	\begin{enumerate}[label=(\alph*)]
	\setlength\itemsep{0em}
	\item $F(x; y) $ = $ \bar{x} \cdot 1 + \bar{y} \cdot 1 $ = $ \bar{x}(y + \bar{y}) + y(x+\bar{x})$ = $ xy + \bar{x}y + \bar{x}y + \bar{x}\bar{y}$ =  $\bar{x}y + \bar{x}\bar{y} + xy$
	\item $x\bar{y}$ is already in sum-of-products form.
	\\Thus,sum-of-products expansion of $x\bar{y}$ is    $x\bar{y}$ itself.
	\item $F(x; y) $ = $ 1 \cdot 1$ = $  (y + \bar{y}) + (x+\bar{x})$ = $ (x+\bar{x})y + (x+\bar{x})\bar{y} $	 =  $ xy + \bar{x}y + x\bar{y} +  \bar{x}\bar{y}$
	\item $F(x; y)$  =  $\bar{y} \cdot 1  $ = $ \bar{y} (x+\bar{x}) $ = $x\bar{y} + \bar{x}\bar{y} $
	\end{enumerate}
	\end{frame}
	

	%----------------------------------------------------------------------------------
	\begin{frame}
	\frametitle{Problem 3 (12.3 Exercise 4)}
	\fontsize{12}{16pt}\selectfont
	Ans: $ \overline{\bar{x}yz} (\bar{x}+y+\bar{z})$
	\end{frame}

	%----------------------------------------------------------------------------------
	\begin{frame}
	\frametitle{Problem 4 (12.4 Exercise 2)}
	\fontsize{12}{16pt}\selectfont
	\begin{enumerate}[label=(\alph*)]
	\setlength\itemsep{0em}
	\item $ xy + \bar{x}y + \bar{x}\bar{y}$
	\item $ xy + x\bar{y}$
	\item $ xy + x\bar{y} + \bar{x}y + \bar{x}\bar{y}$
	\end{enumerate}
	\end{frame}
	
	%----------------------------------------------------------------------------------
	
	%----------------------------------------------------------------------------------


		
%----------------------------------------------------------------------------------
\section{完成作業小時數}
%----------------------------------------------------------------------------------

%----------------------------------------------------------------------------------	
\begin{frame}
\frametitle{完成作業小時數}
\centerline{\fontsize{16}{16pt}\selectfont{\color{blue}完成作業小時數:\color{red}共\underline{  2   }小時\color{blue}(估算,取整數)}}	
\end{frame}
%----------------------------------------------------------------------------------	

%----------------------------------------------------------------------------------
\end{document}
%----------------------------------------------------------------------------------

