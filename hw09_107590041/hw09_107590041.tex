\documentclass[12pt,hyperref={bookmarks=false}]{beamer}

%\usepackage[no-math]{fontspec} % https://www.ptt.cc/bbs/LaTeX/M.1505577244.A.952.html
\usepackage{xeCJK}
\usepackage{amsfonts}
\usepackage{amssymb}
\usepackage{amsmath}
%\usepackage{musixtex}
\usepackage{enumitem}
\usepackage{algorithm,algorithmic}


\renewcommand\qedsymbol{$\blacktriangleleft$}

%\setCJKmainfont{新細明體}
\setCJKmainfont{標楷體}

\usetheme{Malmoe}
\usecolortheme{dolphin}
\usefonttheme[onlymath]{serif}
%\usefonttheme{serif} % professionalfonts
% default, serif, professionalfonts, structurebold, structureitalicserif, structuresmallcapsserif
\useoutertheme{miniframes} %{infolines}
\usepackage{xmpmulti}

\linespread{1.2}

%\newenvironment{num}
%{\leftmargini=6mm\leftmarginii=8mm\begin{itemize}}
%{\end{itemize}}

%\newenvironment{enu}
%{\leftmargini=6mm\leftmarginii=8mm\begin{enumerate}}
%{\end{enumerate}}

%-------------------------------------------------------------
\title{離散數學 107-2}
\subtitle{Homework 09}
\author{姓名: 葉子瑄。學號: 107590041}
\date{Handout: 2019.06.27 (week-15)}

%-------------------------------------------------------------
% page number
% https://tex.stackexchange.com/questions/137022/how-to-insert-page-number-in-beamer-navigation-bars
%-------------------------------------------------------------

\setbeamertemplate{navigation symbols}
{ \insertslidenavigationsymbol 
  \insertframenavigationsymbol   
  \insertsubsectionnavigationsymbol  
  \insertsectionnavigationsymbol
  \insertdocnavigationsymbol  
  \insertbackfindforwardnavigationsymbol 
  \hspace{1em}  
  \usebeamerfont{footline} 
  \insertframenumber
  ~-~
  \inserttotalframenumber
}
\setcounter{page}{1} 
\pagenumbering{arabic} 


%----------------------------------------------------------------------------------
\begin{document}
%----------------------------------------------------------------------------------

\begin{frame}
\titlepage
\end{frame}

\raggedright

\begin{frame}
% \tiny \scriptsize \footnotesize \small \normalsize \large \Large \LARGE \huge \Huge
%\normalsize
\footnotesize
\tableofcontents
\end{frame}
	
%----------------------------------------------------------------------------------
\section{題目}
%----------------------------------------------------------------------------------

    %----------------------------------------------------------------------------------
	\subsection{題目與注意事項}
	%----------------------------------------------------------------------------------
	
	%----------------------------------------------------------------------------------
	\begin{frame}
	\frametitle{Homework 09 題目}
	\fontsize{10pt}{11pt}\selectfont
	\setlength{\baselineskip}{5pt}
	\begin{columns}
	\begin{column}{0.68\textwidth}
	\begin{enumerate}[label=(Prob. \arabic*)]
	\setlength\itemsep{0em}
	\item page 608, chapter 9.1 Exercise 6
	\item page 619, chapter 9.2 Exercise 2
	\item page 627, chapter 9.3 Exercise 14
	\item page 638, chapter 9.4 Exercise 20
	\item page 647, chapter 9.5 Exercise 24
	\item page 662, chapter 9.6 Exercise 8
	\end{enumerate}
	\end{column}
	
%	\begin{column}{0.45\textwidth}
%	\begin{enumerate}[label=(Prob. \arabic*)]
%	\addtocounter{enumi}{21}
%	\setlength\itemsep{0em}
%	\item page xx, chapter xx Exercises xx	
%	\end{enumerate}
%	\end{column}
	
	\end{columns}
	\end{frame}
	
	%----------------------------------------------------------------------------------
	\begin{frame}
	\frametitle{注意事項}
	\fontsize{10}{10pt}\selectfont
	\begin{enumerate}[label=(\alph*)]
	\item 要熟悉 LaTeX 請翻閱 \ \htmladdnormallink{\color{blue}lshort}{https://ctan.org/tex-archive/info/lshort/}。
	\item 記得在最後一頁,回報\selectfont \color{red}{完成作業小時數}(估算,取整數)\selectfont \color{black}{。}
	\item 將檔案夾命名為 \texttt{hw09\_107820xxx},將檔案夾壓縮成 \texttt{hw09\_107820xxx.zip},上傳到\htmladdnormallink{\color{blue}網路學園}{http://elearning.ntut.edu.tw}。
	\item LaTeX 數學符號請查此表: \ \htmladdnormallink{\color{blue}List of LaTeX mathematical symbols}{https://oeis.org/wiki/List_of_LaTeX_mathematical_symbols}。
	\item 作業抄襲,以零分計。作業提供給他人抄襲,以零分計。
	\item 作業遲交一週內成績打五折,作業遲交超過一週以零分計。
	\end{enumerate}
	\end{frame}

%----------------------------------------------------------------------------------
\section{作答區}
%----------------------------------------------------------------------------------

    %----------------------------------------------------------------------------------
	\subsection{解題}
	%----------------------------------------------------------------------------------

	%----------------------------------------------------------------------------------
	\begin{frame}
	\frametitle{Problem 1 (9.1 Exercise 6)}
	\fontsize{10}{10pt}\selectfont
	\begin{enumerate}[label=(\alph*)]
	\setlength\itemsep{0em}
	\item not reflexive, symmetric, not antisymmetric, not transitive.
	\item reflexive, symmetric, not antisymmetric, transitive.	
	\item reflexive , symmetric , not antisymmetric , transitive.
	\item not reflexive , not symmetric , antisymmetric , not transitive.
	\item reflexive , symmetric , not antisymmetric , not transitive.
	\item not reflexive , symmetric, not antisymmetric , not transitive.
	\item not reflexive , not symmetric , antisymmetric , transitive.

	\end{enumerate}
	\end{frame}	%----------------------------------------------------------------------------------
	\begin{frame}
	\frametitle{Problem 2 (9.2 Exercise 2)}
	\fontsize{10}{10pt}\selectfont
	(6, 1, 1, 1), (1, 6, 1, 1), (1, 1, 6, 1), (1, 1, 1, 6), (3, 2, 1, 1), (3, 1, 2, 1), (3, 1, 1, 2), (2, 3, 1, 1), (2, 1, 3, 1),(2, 1, 1, 3), (1, 3, 2, 1), (1, 3, 1, 2), (1, 2, 3, 1), (1, 2, 1, 3), (1, 1, 3, 2), (1, 1, 2, 3)\\
	\end{frame}
	

	%----------------------------------------------------------------------------------
	\begin{frame}
	\frametitle{Problem 3 (9.3 Exercise 14)}
	\fontsize{12}{14pt}\selectfont

	\begin{columns}
	\begin{column}{0.4\textwidth}
	\begin{enumerate}[label=(\alph*)]
	\setlength\itemsep{1em}
	\item
	$
	\begin{bmatrix}
	   0 & 1 & 0 \\
 	   1 & 1 & 1 \\
 	   1 & 1 & 1
	\end{bmatrix}
	$
	\item
	$
	\begin{bmatrix}
	   0 & 1 & 0 \\
 	   0 & 1 & 1 \\
 	   1 & 0 & 0
	\end{bmatrix}
	$
	\item
	$
	\begin{bmatrix}
	   0 & 1 & 1 \\
 	   1 & 1 & 1 \\
 	   0 & 1 & 0
	\end{bmatrix}
	$
	\end{enumerate}	\end{column}

	\begin{column}{0.4\textwidth}
	\begin{enumerate}[label=(\alph*)]
	\addtocounter{enumi}{3}
	\setlength\itemsep{1em}
	\item
	$
	\begin{bmatrix}
	   1 & 1 & 1 \\
 	   1 & 1 & 1 \\
 	   0 & 1 & 0
	\end{bmatrix}
	$
	\item
	$
	\begin{bmatrix}
	   0 & 0 & 0 \\
 	   1 & 0 & 0 \\
 	   0 & 1 & 1
	\end{bmatrix}
	$
	\end{enumerate}
	\end{column}	
	\end{columns}
	\end{frame}

	%----------------------------------------------------------------------------------
	\begin{frame}
	\frametitle{Problem 4 (9.4 Exercise 20)}
	\fontsize{10}{10pt}\selectfont
	\begin{enumerate}[label=(\alph*)]
	\setlength\itemsep{1em}
	\item The pair $(a, b)$ is in $R^2$ when it is possible to fly from $a$ to $b$ with a scheduled stop in some intermediate city.
	\item The pair $(a, b)$ is in $R^3$ when it is possible to fly from $a$ to $b$ with two scheduled stop in some intermediate cities.
	\item The pair $(a, b)$ is in $R^*$ precisely when it is possible to fly from $a$ to $b$.
	\end{enumerate}
	\end{frame}
	
	%----------------------------------------------------------------------------------
	\begin{frame}
	\frametitle{Problem 5 (9.5 Exercise 24)}
	\fontsize{10}{10pt}\selectfont
	(是非題)
	\vspace*{0.3cm}
	\begin{enumerate}[label=(\alph*)]
	\setlength\itemsep{1em}
	\item This is not an equivalence relation.
	\item This is an equivalence relation.
	\item This is an equivalence relation.
	\end{enumerate}
	\end{frame}
	
    %----------------------------------------------------------------------------------
	\begin{frame}
	\frametitle{Problem 6 (9.6 Exercise 8)}
	\fontsize{10}{10pt}\selectfont
	(是非題)
	\vspace*{0.3cm}
	\begin{enumerate}[label=(\alph*)]
	\setlength\itemsep{1em}
	\item not a partial order
	\item partial order
	\item not a partial order
	\end{enumerate}
	\end{frame}
	
	%----------------------------------------------------------------------------------

		
%----------------------------------------------------------------------------------
\section{完成作業小時數}
%----------------------------------------------------------------------------------

%----------------------------------------------------------------------------------	
\begin{frame}
\frametitle{完成作業小時數}
\centerline{\fontsize{16}{16pt}\selectfont{\color{blue}完成作業小時數:\color{red}共\underline{  4   }小時\color{blue}(估算,取整數)}}	
\end{frame}
%----------------------------------------------------------------------------------	

%----------------------------------------------------------------------------------
\end{document}
%----------------------------------------------------------------------------------

