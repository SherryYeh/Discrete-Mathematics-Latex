\documentclass[14pt,hyperref={bookmarks=false}]{beamer}

%\usepackage[no-math]{fontspec} % https://www.ptt.cc/bbs/LaTeX/M.1505577244.A.952.html
\usepackage{xeCJK}
\usepackage{amsfonts}
\usepackage{amssymb}
\usepackage{amsmath}
%\usepackage{musixtex}
\usepackage{enumitem}
\usepackage{algorithm,algorithmic}


\renewcommand\qedsymbol{$\blacktriangleleft$}

%\setCJKmainfont{新細明體}
\setCJKmainfont{標楷體}

\usetheme{Malmoe}
\usecolortheme{dolphin}
\usefonttheme[onlymath]{serif}
%\usefonttheme{serif} % professionalfonts
% default, serif, professionalfonts, structurebold, structureitalicserif, structuresmallcapsserif
\useoutertheme{miniframes} %{infolines}
\usepackage{xmpmulti}

\linespread{1.2}

%\newenvironment{num}
%{\leftmargini=6mm\leftmarginii=8mm\begin{itemize}}
%{\end{itemize}}

%\newenvironment{enu}
%{\leftmargini=6mm\leftmarginii=8mm\begin{enumerate}}
%{\end{enumerate}}

%-------------------------------------------------------------
\title{離散數學 107-2}
\subtitle{Homework 04}
\author{姓名: 葉子瑄。學號: 107590041}
\date{截止收件: 2019.05.08 (Wednesday) 23:59 pm \\ (week-12)}

%-------------------------------------------------------------
% page number
% https://tex.stackexchange.com/questions/137022/how-to-insert-page-number-in-beamer-navigation-bars
%-------------------------------------------------------------

\setbeamertemplate{navigation symbols}
{ \insertslidenavigationsymbol 
  \insertframenavigationsymbol   
  \insertsubsectionnavigationsymbol  
  \insertsectionnavigationsymbol
  \insertdocnavigationsymbol  
  \insertbackfindforwardnavigationsymbol 
  \hspace{1em}  
  \usebeamerfont{footline} 
  \insertframenumber
  ~-~
  \inserttotalframenumber
}
\setcounter{page}{1} 
\pagenumbering{arabic} 


%----------------------------------------------------------------------------------
\begin{document}
%----------------------------------------------------------------------------------

\begin{frame}
\titlepage
\end{frame}

\raggedright

\begin{frame}
% \tiny \scriptsize \footnotesize \small \normalsize \large \Large \LARGE \huge \Huge
%\normalsize
\footnotesize
\tableofcontents
\end{frame}
	
%----------------------------------------------------------------------------------
\section{題目}
%----------------------------------------------------------------------------------

    %----------------------------------------------------------------------------------
	\subsection{題目與注意事項}
	%----------------------------------------------------------------------------------
	
	%----------------------------------------------------------------------------------
	\begin{frame}
	\frametitle{Homework 04 題目}
	\fontsize{11pt}{12pt}\selectfont
	\setlength{\baselineskip}{5pt}
	\begin{columns}
	\begin{column}{0.68\textwidth}
	\begin{enumerate}[label=(Prob. \arabic*)]
	\setlength\itemsep{0em}
	\item page 259, chapter 4.1 Exercise 30
	\item page 269, chapter 4.2 Exercise 4
	\item page 290, chapter 4.3 Exercise 40(c)
	\item page 301, chapter 4.4 Exercise 6(b)
	\item page 301, chapter 4.4 Exercise 20	
	\item page 308, chapter 4.5 Exercise 2
	\item page 323, chapter 4.6 Example 26
	\end{enumerate}
	\end{column}
	
%	\begin{column}{0.45\textwidth}
%	\begin{enumerate}[label=(Prob. \arabic*)]
%	\addtocounter{enumi}{21}
%	\setlength\itemsep{0em}
%	\item page xx, chapter xx Exercises xx	
%	\end{enumerate}
%	\end{column}
	
	\end{columns}
	\end{frame}
	
	%----------------------------------------------------------------------------------
	\begin{frame}
	\frametitle{注意事項}
	\fontsize{10}{10pt}\selectfont
	\begin{enumerate}[label=(\alph*)]
	\item 要熟悉 LaTeX 請翻閱 \ \htmladdnormallink{\color{blue}lshort}{https://ctan.org/tex-archive/info/lshort/}。
	\item 記得在最後一頁,回報\selectfont \color{red}{完成作業小時數}(估算,取整數)\selectfont \color{black}{。}
	\item 將檔案夾命名為 \texttt{hw04\_107590xxx},將檔案夾壓縮成 \texttt{hw04\_107590xxx.zip},上傳到\htmladdnormallink{\color{blue}網路學園}{http://elearning.ntut.edu.tw}。
	\item LaTeX 數學符號請查此表: \ \htmladdnormallink{\color{blue}List of LaTeX mathematical symbols}{https://oeis.org/wiki/List_of_LaTeX_mathematical_symbols}。
	\item 作業抄襲,以零分計。作業提供給他人抄襲,以零分計。
	\item 作業遲交一週內成績打五折,作業遲交超過一週以零分計。
	\end{enumerate}
	\end{frame}

%----------------------------------------------------------------------------------
\section{作答區}
%----------------------------------------------------------------------------------

    %----------------------------------------------------------------------------------
	\subsection{解題}
	%----------------------------------------------------------------------------------
	
	%----------------------------------------------------------------------------------
	\begin{frame}
	\frametitle{Problem 01 (4.1 Exercise 30)}
	\fontsize{10}{10pt}\selectfont
	\begin{enumerate}[label=(\alph*)]
	\item $-3 \equiv 43~ (\text{mod}~ 23) $
	\item $-12 \equiv 17~ (\text{mod}~ 29) $
	\item $94 \equiv -11~ (\text{mod}~ 21) $
	\end{enumerate}
	\end{frame}
	
	%----------------------------------------------------------------------------------
	\begin{frame}
	\frametitle{Problem 02 (4.2 Exercise 4)}
	\fontsize{10}{10pt}\selectfont
	\begin{enumerate}[label=(\alph*)]
	\item $27$
	\item $693$
	\item $958$
	\item $31775$
	\end{enumerate}
	\end{frame}
	
	%----------------------------------------------------------------------------------
	\begin{frame}
	\frametitle{Problem 03 (4.3 Exercise 40(c))}
	\fontsize{9}{10pt}\selectfont
	\setlength{\baselineskip}{5pt}

	(a)	The steps used by the Euclidean algorithm to find $\text{gcd}(35, 78)$ are
	\begin{eqnarray*}
	78 & = & 2 · 35 + 8 \\
	35 & = & 4 · 8 + 3 \\
	8 & = & 2 · 3 + 2  \\
	3 & = & 1 · 2 + 1 \\
	2 & = & 2 · 1
	\end{eqnarray*}

    \vspace*{0.1cm}
    
    (b) Then we need to work our way back up
	\begin{eqnarray*}
	1 & = & 3 − 2                  \\
	~ & = & 3 − (8 - 2 · 3)        =  3 · 3 − 8 \\
	~ & = & 3 · (35 - 4 · 8) -8 =  3 · 35 − 13 · 8 \\
	~ & = & 3 · 35 − 13 · (78 - 2 · 35)  =  29 · 35 − 13 · 78
	\end{eqnarray*}
	\end{frame}
	%----------------------------------------------------------------------------------
	\begin{frame}
	\frametitle{Problem 04 (4.4 Example 6(b))}
	\fontsize{7}{8pt}\selectfont
	\setlength{\baselineskip}{5pt}

	(a)	First we go through the Euclidean algorithm computation that $gcd(34, 89) = 1$:
	\begin{eqnarray*}
	89 & = & 2 · 34 + 21 \\
	34 & = & 1 · 21 + 13  \\
	21 & = & 1 · 13 + 8  \\
	13 & = & 1 · 8 + 5  \\
	 8 & = & 1 · 5 + 3  \\
	 5 & = & 1 · 3 + 2  \\
	 3 & = & 1 · 2 + 1 \\
	 2 & = & 2 · 1 
	\end{eqnarray*}


    (b) Then we reverse our steps and write 1 as the desired linear combination:
    
	\begin{eqnarray*}
	1 & = & 3 − 2 \\
	~ & = & 3 - (5 - 3) = 2 · 3 - 5  \\
	~ & = & 2 · (8 - 5) - 5 = 2 · 8 - 3 · 5  \\
	~ & = & 2 · 8 - 3 · (13 - 8) = 5 · 8 - 3 · 13  \\
	~ & = & 5 · (21 - 13) - 3 · 13 = 5 · 21 - 8 · 13  \\
	~ & = & 5 · 21 - 8 · (34 - 21) = 13 · 21 - 8 · 34  \\
	~ & = & 13 · (89 - 2 · 34) - 8 · 34 = 13 · 89 - 34 · 34
	\end{eqnarray*}
	
    Thus s = -34, so an inverse of 34 modulo 89 is -34, which can also be written as 55.
	\end{frame}
	
	%----------------------------------------------------------------------------------
	\begin{frame}
	\frametitle{Problem 05 (4.4 Exercises 20)}
	\fontsize{10}{10pt}\selectfont
	The answer will be unique modulo $3 · 4 · 5 = 60$.\\
	$a_1 = 2, m_1 = 3$\\ 
	$a_2 = 1, m_2 = 4$\\ 
	$a_3 = 3, m_3 = 5$\\ 
    \vspace*{0.1cm}
	$m = m_1 · m_2 · m_3 =60$ \\
    $M_1 = 60/3 = 20$\\
    $M_2 = 60/4 = 15$\\
    $M_3 = 60/5 = 12$\\
    \vspace*{0.1cm}
	Then we need to find inverses $y_i$ of $M_i$ modulo $m_i$\\
    $y_1 = 2$\\
    $y_2 = 3$\\
    $y_3 = 3$\\
    $ x =  a_1 M_1 y_1 + a_2 M_2 y_2 + a_3 M_3 y_3 = 233 \equiv 53 \pmod{60}$\\
    So the solutions are all integers of the form $53 + 60k$, where $k$ is an integer.
	\end{frame}
	

	%----------------------------------------------------------------------------------
	\begin{frame}
	\frametitle{Problem 06 (4.5 Exercises 2)}
	\fontsize{10}{10pt}\selectfont
	\begin{enumerate}[label=(\alph*)]
	\item $58$
	\item $60$
	\item $52$
	\item $3$
	\end{enumerate}
	\end{frame}

	%----------------------------------------------------------------------------------
	\begin{frame}
	\frametitle{Problem 7 (4.6 Exercises 22)}
	\fontsize{11}{12pt}\selectfont
	First we find $d = 2753$, the inverse of $e = 17~modulo~52 · 60$. \\
 
    \vspace*{0.3cm}
    
	Next we compute $c^d \pmod{n}$ for each of the four given numbers:\\ $3185^{2753} \pmod{3233} = 1816$ (which are the letters SQ),\\
$2038^{2753} \pmod{3233} = 2008$ (which are the letters UI),\\
$2460^{2753} \pmod{3233} = 1717$ (which are the letters RR), and \\
$2550^{2753} \pmod{3233} = 0411$ (which are the letters EL).\\

    \vspace*{0.3cm}
    
    The message is SQUIRREL.
	\end{frame}
	%----------------------------------------------------------------------------------

		
%----------------------------------------------------------------------------------
\section{完成作業小時數}
%----------------------------------------------------------------------------------

%----------------------------------------------------------------------------------	
\begin{frame}
\frametitle{完成作業小時數}
\centerline{\fontsize{16}{16pt}\selectfont{\color{blue}完成作業小時數:\color{red}共\underline{   3   }小時\color{blue}(估算,取整數)}}	
\end{frame}
%----------------------------------------------------------------------------------	

%----------------------------------------------------------------------------------
\end{document}
%----------------------------------------------------------------------------------

