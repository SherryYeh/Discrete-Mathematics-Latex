\documentclass[14pt,hyperref={bookmarks=false}]{beamer}

%\usepackage[no-math]{fontspec} % https://www.ptt.cc/bbs/LaTeX/M.1505577244.A.952.html
\usepackage{xeCJK}
\usepackage{amsfonts}
\usepackage{amssymb}
\usepackage{amsmath}
%\usepackage{musixtex}
\usepackage{enumitem}

\renewcommand\qedsymbol{$\blacktriangleleft$}

%\setCJKmainfont{新細明體}
\setCJKmainfont{標楷體}

\usetheme{Malmoe}
\usecolortheme{dolphin}
\usefonttheme[onlymath]{serif}
%\usefonttheme{serif} % professionalfonts
% default, serif, professionalfonts, structurebold, structureitalicserif, structuresmallcapsserif
\useoutertheme{miniframes} %{infolines}
\usepackage{xmpmulti}

\linespread{1.2}

%\newenvironment{num}
%{\leftmargini=6mm\leftmarginii=8mm\begin{itemize}}
%{\end{itemize}}

%\newenvironment{enu}
%{\leftmargini=6mm\leftmarginii=8mm\begin{enumerate}}
%{\end{enumerate}}

%-------------------------------------------------------------
\title{離散數學 107-2}
\subtitle{Homework 01}
\author{姓名: 葉子瑄。學號: 107590041}
\date{截止收件: 2019.03.20 (Wednesday) 23:59 pm Week-05}

%-------------------------------------------------------------
% page number
% https://tex.stackexchange.com/questions/137022/how-to-insert-page-number-in-beamer-navigation-bars
%-------------------------------------------------------------

\setbeamertemplate{navigation symbols}
{ \insertslidenavigationsymbol 
  \insertframenavigationsymbol   
  \insertsubsectionnavigationsymbol  
  \insertsectionnavigationsymbol
  \insertdocnavigationsymbol  
  \insertbackfindforwardnavigationsymbol 
  \hspace{1em}  
  \usebeamerfont{footline} 
  \insertframenumber
  ~-~
  \inserttotalframenumber
}
\setcounter{page}{1} 
\pagenumbering{arabic} 


%----------------------------------------------------------------------------------
\begin{document}
%----------------------------------------------------------------------------------

\begin{frame}
\titlepage
\end{frame}

\raggedright

\begin{frame}
% \tiny \scriptsize \footnotesize \small \normalsize \large \Large \LARGE \huge \Huge
%\normalsize
\footnotesize
\tableofcontents
\end{frame}
	
%----------------------------------------------------------------------------------
\section{題目}
%----------------------------------------------------------------------------------

    %----------------------------------------------------------------------------------
	\subsection{題目與注意事項}
	%----------------------------------------------------------------------------------
	
	%----------------------------------------------------------------------------------
	\begin{frame}
	\frametitle{Homework 01 題目}
	\fontsize{6pt}{7pt}\selectfont
	\setlength{\baselineskip}{5pt}
	\begin{columns}
	\begin{column}{0.49\textwidth}
	\begin{enumerate}[label=(Prob. \arabic*)]
	\setlength\itemsep{0em}
	\item page 16, chapter 1.1 Exercise 34(e)小題
	\item page 24, chapter 1.2 Exercise 8
	\item page 38, chapter 1.3 Exercise 10(c)小題
	\item page 58, chapter 1.4 Exercise 28
	\end{enumerate}
	\end{column}
	
	\begin{column}{0.49\textwidth}
	\begin{enumerate}[label=(Prob. \arabic*)]
	\addtocounter{enumi}{4}
	\setlength\itemsep{0em}
	\item page 69, chapter 1.5 Exercise 10
	\item page 82, chapter 1.6 Exercise 6
	\item page 96, chapter 1.7 Exercise 30
	\item page 113, chapter 1.8 Exercise 6
	\end{enumerate}
	\end{column}
	
	\end{columns}
	\end{frame}
	
	%----------------------------------------------------------------------------------
	\begin{frame}
	\frametitle{注意事項}
	\fontsize{10}{10pt}\selectfont
	\begin{enumerate}[label=(\alph*)]
	\item 要熟悉 LaTeX 請翻閱 \ \htmladdnormallink{\color{blue}lshort}{https://ctan.org/tex-archive/info/lshort/}。
	\item 記得在最後一頁,回報\selectfont \color{red}{完成作業小時數}(估算,取整數)\selectfont \color{black}{。}
	\item 將檔案夾命名為 \texttt{hw01\_107820xxx},將檔案夾壓縮成 \texttt{hw01\_107820xxx.zip},上傳到\htmladdnormallink{\color{blue}網路學園}{http://elearning.ntut.edu.tw}。
	\item LaTeX 數學符號請查此表: \ \htmladdnormallink{\color{blue}List of LaTeX mathematical symbols}{https://oeis.org/wiki/List_of_LaTeX_mathematical_symbols}。
	\item 作業抄襲,以零分計。作業提供給他人抄襲,以零分計。
	\item 作業遲交一週內成績打五折,作業遲交超過一週以零分計。
	\end{enumerate}
	\end{frame}

%----------------------------------------------------------------------------------
\section{作答區}
%----------------------------------------------------------------------------------

    %----------------------------------------------------------------------------------
	\subsection{解題}
	%----------------------------------------------------------------------------------
	
	%----------------------------------------------------------------------------------
	\begin{frame}
	\frametitle{Problem 1 (1.1 Exercise 34(e)小題)}
	\fontsize{10}{10pt}\selectfont
	% https://www.tablesgenerator.com/
	\begin{table}[]
	\centering
	\caption{Truth Table for the Compound Propositions}
	\label{t1}
	\begin{tabular}{c|c|c|c|c|c}
	\hline
	 $p$ & $q$ & $\neg p$ & $(q \rightarrow \neg p)$ & $(p \leftrightarrow q)$ & $ (q \rightarrow \neg p) \leftrightarrow (p \leftrightarrow q)$  \\ \hline
	 T  & T & F & F & T & F \\ % 作答後,記得移除 作答 二字。
	 T  & F & F & T & F & F \\ 
	 F  & T & T & T & F & F \\ 
	 F  & F & T & T & T & T \\ 
	\hline
	\end{tabular}
	\end{table}

	\end{frame}
	
	%----------------------------------------------------------------------------------
	\begin{frame}
	\frametitle{Problem 2 (1.2 Exercise 8)}
	\fontsize{10}{10pt}\selectfont
	\begin{enumerate}[label=(\alph*)]
	\item $r \land \neg p$
	\item $q \rightarrow (p \land r)$ % 作答後,記得移除 作答 二字。
	\item $\neg q \rightarrow \neg r$
	\item $(\neg p \land r) \rightarrow q$
	\end{enumerate}
	\end{frame}

	%----------------------------------------------------------------------------------
	\begin{frame}
	\frametitle{Problem 3 (1.3 Exercise 10(c)小題)}
	\fontsize{10}{10pt}\selectfont
	\begin{proof}
	\begin{eqnarray*}
	  (p \rightarrow \neg q) \rightarrow (\neg p \rightarrow q) & \equiv & \neg (p \rightarrow \neg q) \lor (\neg p \rightarrow q ) \\
	  & \equiv & \neg (\neg p \lor \neg q) \lor (\neg p \rightarrow q  ) \\ % 作答後,記得移除 \text{作答} 。
	  & \equiv & (p \land q) \lor( p \lor q ) \\
	  & \equiv & (p \lor q ) \\
	  & \equiv & \neg p \rightarrow q
	\end{eqnarray*}
	\end{proof}
	\end{frame}
	
	%----------------------------------------------------------------------------------
	\begin{frame}
	\frametitle{Problem 4 (1.4 Exercise 28)}
	\fontsize{10}{10pt}\selectfont
	let $R(x)$ be “$x$ is in the correct place,”\\
    let $E(x)$ be “$x$ is in excellent condition,”\\
    let $T(x)$ be “$x$ is a [or your] tool,”\\ and let the domain of discourse be all things.
	\begin{enumerate}[label=(\alph*)]
	\item $\exists x ~ \neg R(x)$
	\item $\forall x ~ (R(x) \land E(x)) $ % 作答後,記得移除 \text{作答} 。
	\item $\forall x ~ T(x)(R(x) \land E(x)) $
	\item $\exists x ~( \neg(R(x) \land E(x)) $
	\item $\neg \exists x ~ T(x)(R(x)\land \neg E(x)) $
	\end{enumerate}
	\end{frame}

	%----------------------------------------------------------------------------------
	\begin{frame}
	\frametitle{Problem 5 (1.5 Exercise 10)}
	\fontsize{10}{10pt}\selectfont
	\begin{enumerate}[label=(\alph*)]
	\item $\forall x ~F(x, \textrm{Fred})$ 
	\item $\forall y ~F(\textrm{Evelyn},y} $ % 作答後,記得移除 \text{作答} 。
	\item $ \forall x ~ \exists y ~ F(x,y) $ 
	\item $ \neg \exists x ~ \forall y ~ F(x,y) $ 
	\item $ \exists x ~ \forall y ~ F(x,y) $ 
	\item $ \neg \exists x ~ (F(x,\textrm{Fred})\land F(x,\textem{Jerry}))$
	\item $ \exists y ~ \exists z ~ (F(\textrm{Nancy},y)\land (F(\textrm{Nancy},z)\land x \ne y \land\forall w ~ (F(\textrm{Nancy},w)\rightarrow(w=y \lor w=z))) $ 
	\item $ \exists y ~ (\forall x ~ F(x,y)\land (\forall z ~((\forall w~F(w,z))\rightarrow z=y)) $ 
	\item $ \forall x ~ \neg F(x,x)  $ 
	\item $ \exists x ~ \exists y ~ (F(x,y)\land(\forall z ~ (F(x,z)\rightarrow z=y \lor z=x ))$ 
	\end{enumerate}
	\end{frame}

	%----------------------------------------------------------------------------------
	\begin{frame}
	\frametitle{Problem 6 (1.6 Exercise 6)}
	\fontsize{8}{8pt}\selectfont
	
let $r$ be the proposition "It rains," \\
let $f$ be the proposition "It is foggy," \\
let $s$ be the proposition "The sailing race will be held," \\
let $l$ be the proposition "The life saving demonstration will go on," and \\
let $t$ be the proposition "The trophy will be awarded." \\

	\begin{table}[]
	\centering
	\begin{tabular}{|c|l|l|}
	\hline
	% 作答後,記得移除 \text{作答} 。
 Step & 推導                 & Reason \\ \hline
	1 & $ \neg t $           & Hypothesis \\
	2 & $ s \rightarrow t $  & Hypothesis \\
	3 & $ \text{} $       & Modus tollens using (1) and (2) \\
	4 & $ (\neg r \lor \neg f) \rightarrow (s \land l)$ & Hypothesis \\
	5 & $ \neg (s \land l)\rightarrow\neg(\neg r \lor \neg f) $  & Contrapositive of (4) \\
	6 & $ \(neg s \lor \neg l)\rightarrow(r \land f) $  & De Morgan's law and double negative \\
	7 & $ \(\neg s \lor \neg l) $  & Addition, using (3) \\
	8 & $ \(r \land f) $  & Modus ponens using (6) and (7) \\
	9 & $ \r $  & Simplification using (8) \\ \hline
	\end{tabular}
	\end{table}
	\end{frame}


	%----------------------------------------------------------------------------------
	\begin{frame}
	\frametitle{Problem 7 (1.7 Exercise 30)}
	\fontsize{10}{10pt}\selectfont
	
	\begin{columns}
	
	\begin{column}{0.48\textwidth}
	For the "if" part, there are two cases.\\
	If $m = n$ \\
	$m - n = 0$ \\ % 作答後,記得移除 作答 二字。
	If $m = -n$\\
	$m + n = 0$ \\
	\end{column}
	
	\begin{column}{0.45\textwidth}
	For the "only if" part, we suppose that $m^2 = n^2$.\\
	$(m - n) (m + n) = 0$ \\
	\end{column}
	\end{columns}
	
	\end{frame}


	%----------------------------------------------------------------------------------
	\begin{frame}
	\frametitle{Problem 8 (1.8 Exercise 6)}
	\fontsize{10}{10pt}\selectfont
    題目: Use a proof by cases to show that $\text{min}(a, \text{min}(b, c)) = \text{min}(\text{min}(a, b), c)$ whenever $a$, $b$, and $c$ are real numbers.
	
	%作答後請移除前述 請作答 這三個字
	\begin{proof}
	Case 1: If a is smallest.\\
	$ a \le \text{min(b,c)) , a \le b , a \le c $ \\ % 作答後,記得移除 作答 二字。
	$ \text{min}(a,b)= a$\\
	$ a = \text{min}(\text{min}(a,b),c)=\text{min}(a,c)=a $\\
	Case 2: If a is smallest.\\
	$ b \le \text{min(b,c)) , b \le b , b \le c $ 
	$ \text{min}(a,b)= b$\\
	$ b = \text{min}(\text{min}(a,b),c)=\text{min}(b,c)=b $\\
	Case 3: If c is smallest.\\
	$ c \le \text{min(b,c)) , c \le b , c \le c $ 
	$ c \le \text{min}(a,b)$\\
	$ c = \text{min}(\text{min}(a,b),c)=c $\\
	\end{proof}
	\end{frame}
	
	%----------------------------------------------------------------------------------

		
%----------------------------------------------------------------------------------
\section{完成作業小時數}
%----------------------------------------------------------------------------------

%----------------------------------------------------------------------------------	
\begin{frame}
\frametitle{完成作業小時數}
\centerline{\fontsize{16}{16pt}\selectfont{\color{blue}完成作業小時數:\color{red}共\underline{   5   }小時\color{blue}(估算,取整數)}}	
\end{frame}
%----------------------------------------------------------------------------------	

%----------------------------------------------------------------------------------
\end{document}
%----------------------------------------------------------------------------------

